\subsection{Suggested Workflows and Strategies}

I write on the web, and I also write on my personal computer and would like to keep everything synchronised. Plus, I write with my colleagues and my students. In the past, any/all of these happened:

\begin{enumerate}
\item I'd write in Latex and my colleagues would write in Word
\item I'd then have to write in Word and send them Word-ised documents and they would return me back their written comments
\item I'd then have to read or open them again in my word processing app/software and rewrite and resend the documents
\item In course of time, a log of documents would gather, the latest document would be the one on which we'd work.
\item In course of time, the last document would be the one would be sent around to multiple authors or to my colleagues and we'd work on this
\item The codes would reside elsewhere, the data may be in the same folder but certainly not in the context of the same file. 
\item The citation and reference management was in Endnote, and I used to use BibTeX but not all my colleagues would be using BibTeX and they'd be familiar with Endnote. While Endnote works very well, sharing data across with people who do and do not use Endnote can sometimes be problematic and shows up errors. How about using one referencing file and the writing process itself takes care of the referencing and citation management issue?
\end{enumerate}

Based on my past experience of working like this, I know that this can create a lot of documents to manage. On the other hand, I'd rather prefer we have one document, and we write using simple plain text, not worry about additional text font management and style management (we most often do not do them anyway), but still have adequate control over what we write and our comments. I personally do not care much for strikeout texts, and additional marginal comments. I'd rather have the comments written directly on the body of the text as these are all reasonable remarks on which we could work. 

So, the strategy note is this:

\begin{enumerate}
\item Let's use one large document
\item Use that one large document to write everything
\item The one large document gets expanded, commented, and written and rewritten and saves itself and has version control installed.
\item one document where we also keep our source codes, tables, and data
\item When the time comes, we can just delve into that one document, and carve out our presentations, our papers, our reports and our theses. Sometimes, it is only going to be a presentation; at other times, it is going to be a journal article. Yet at another instance, it will be only a report. But it is going to be one source from where we create all of these. 
\end{enumerate}

\subsubsection{So in summary:}

\begin{enumerate}
\item The writing happens in plain text (no formatting skills beyond some simple formatting headers in either markdown or LaTeX is needed
\item The data analysis can take place anywhere in any software but the source codes and data are made available to everyone in the author group using CSV file format and the codes in plain text. All statistical data analysis software allows sharing of codes in plain text, so this is not going to be a problem
\item We also use ipython notebooks for sharing of data and outputs (this is optional)
\item graphics and tables are placed in separate text chunks (more on this)
\end{enumerate}


    
    
    
    
    