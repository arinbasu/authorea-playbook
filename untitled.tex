\textit{The Need for this Playbook} 

The purpose of writing this playbook is twofold: 

\begin{enumerate}
\item I have started moving all my academic writing workflow to the web
\item I have started using Authorea for my workflow of academic writing for grants, papers, books, tutorials, and indeed everything else
\item Quite often my colleagues and students wonder where to start and how to use it strategically
\item Partly to answer my own doubts and questions
\item This is more of a manual on Authorea that I will follow
\end{enumerate}

\textbf{Disclaimer}

I have no affiliation with Authorea beyond being a writer here. I hold no stakes and no briefs. This is my playbook for strategies I plan to use with Authorea as a device to write my papers and bring in all my analyses to one place. Also, Auhthorea is a central repository of helping to write papers and collaborative documents. If this goes away, we need to find alternative places to do the work in principle. 

\textbf{What I assume and will not cover in any further}

\begin{itemize}
\item You know how to open an account in Authorea. If not, please visit \href{http://www.authorea.com}{Authorea} website and create accounts to get started. Also, many of you have already received an email from me or your initiators and please follow the instructions there.
\item Writing in plain text or in markdown and in LaTeX. You do not need to worry about writing in Latex if you focus only on writing in Authorea as Authorea does everything for you
\item Third party writing apps. I myself use Scrivener for writing texts and then sync with Authorea (more on this in subsequent sections)
\end{itemize}

This is a document on strategies and usage, not a full manual on using Authorea. Note that this is itself written in Authorea, so if you decide to fork and experiment with it, please feel free to do so.
    